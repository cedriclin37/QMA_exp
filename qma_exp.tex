\documentclass[english]{article}
\usepackage[T1]{fontenc}
\usepackage[latin9]{inputenc}
\usepackage{geometry}
\geometry{verbose,lmargin=1in,rmargin=1in}
\usepackage{amsthm}
\usepackage{amsmath}
\usepackage{amssymb}

\makeatletter
%%%%%%%%%%%%%%%%%%%%%%%%%%%%%% Textclass specific LaTeX commands.
\numberwithin{equation}{section}
\numberwithin{figure}{section}
\theoremstyle{plain}
\newtheorem{thm}{\protect\theoremname}
  \theoremstyle{definition}
  \newtheorem{defn}[thm]{\protect\definitionname}
  \theoremstyle{plain}
  \newtheorem{lem}[thm]{\protect\lemmaname}
  \theoremstyle{definition}
  \newtheorem{problem}[thm]{\protect\problemname}
  \theoremstyle{remark}
  \newtheorem{rem}[thm]{\protect\remarkname}
  \theoremstyle{remark}
  \newtheorem*{rem*}{\protect\remarkname}
  \theoremstyle{plain}
  \newtheorem{algorithm}[thm]{\protect\algorithmname}

\@ifundefined{date}{}{\date{}}
%%%%%%%%%%%%%%%%%%%%%%%%%%%%%% User specified LaTeX commands.
\usepackage{tikz} 

%%    Q-circuit version 2
%    Copyright (C) 2004  Steve Flammia & Bryan Eastin
%    Last modified on: 9/16/2011
%
%    This program is free software; you can redistribute it and/or modify
%    it under the terms of the GNU General Public License as published by
%    the Free Software Foundation; either version 2 of the License, or
%    (at your option) any later version.
%
%    This program is distributed in the hope that it will be useful,
%    but WITHOUT ANY WARRANTY; without even the implied warranty of
%    MERCHANTABILITY or FITNESS FOR A PARTICULAR PURPOSE.  See the
%    GNU General Public License for more details.
%
%    You should have received a copy of the GNU General Public License
%    along with this program; if not, write to the Free Software
%    Foundation, Inc., 59 Temple Place, Suite 330, Boston, MA  02111-1307  USA

% Thanks to the Xy-pic guys, Kristoffer H Rose, Ross Moore, and Daniel Müllner,
% for their help in making Qcircuit work with Xy-pic version 3.8.  
% Thanks also to Dave Clader, Andrew Childs, Rafael Possignolo, Tyson Williams,
% Sergio Boixo, Cris Moore, Jonas Anderson, and Stephan Mertens for helping us test 
% and/or develop the new version.

\usepackage{xy}
\xyoption{matrix}
\xyoption{frame}
\xyoption{arrow}
\xyoption{arc}

\usepackage{ifpdf}
\ifpdf
\else
\PackageWarningNoLine{Qcircuit}{Qcircuit is loading in Postscript mode.  The Xy-pic options ps and dvips will be loaded.  If you wish to use other Postscript drivers for Xy-pic, you must modify the code in Qcircuit.tex}
%    The following options load the drivers most commonly required to
%    get proper Postscript output from Xy-pic.  Should these fail to work,
%    try replacing the following two lines with some of the other options
%    given in the Xy-pic reference manual.
\xyoption{ps}
\xyoption{dvips}
\fi

% The following resets Xy-pic matrix alignment to the pre-3.8 default, as
% required by Qcircuit.
\entrymodifiers={!C\entrybox}

\newcommand{\bra}[1]{{\left\langle{#1}\right\vert}}
\newcommand{\ket}[1]{{\left\vert{#1}\right\rangle}}
    % Defines Dirac notation. %7/5/07 added extra braces so that the commands will work in subscripts.
\newcommand{\qw}[1][-1]{\ar @{-} [0,#1]}
    % Defines a wire that connects horizontally.  By default it connects to the object on the left of the current object.
    % WARNING: Wire commands must appear after the gate in any given entry.
\newcommand{\qwx}[1][-1]{\ar @{-} [#1,0]}
    % Defines a wire that connects vertically.  By default it connects to the object above the current object.
    % WARNING: Wire commands must appear after the gate in any given entry.
\newcommand{\cw}[1][-1]{\ar @{=} [0,#1]}
    % Defines a classical wire that connects horizontally.  By default it connects to the object on the left of the current object.
    % WARNING: Wire commands must appear after the gate in any given entry.
\newcommand{\cwx}[1][-1]{\ar @{=} [#1,0]}
    % Defines a classical wire that connects vertically.  By default it connects to the object above the current object.
    % WARNING: Wire commands must appear after the gate in any given entry.
\newcommand{\gate}[1]{*+<.6em>{#1} \POS ="i","i"+UR;"i"+UL **\dir{-};"i"+DL **\dir{-};"i"+DR **\dir{-};"i"+UR **\dir{-},"i" \qw}
    % Boxes the argument, making a gate.
\newcommand{\meter}{*=<1.8em,1.4em>{\xy ="j","j"-<.778em,.322em>;{"j"+<.778em,-.322em> \ellipse ur,_{}},"j"-<0em,.4em>;p+<.5em,.9em> **\dir{-},"j"+<2.2em,2.2em>*{},"j"-<2.2em,2.2em>*{} \endxy} \POS ="i","i"+UR;"i"+UL **\dir{-};"i"+DL **\dir{-};"i"+DR **\dir{-};"i"+UR **\dir{-},"i" \qw}
    % Inserts a measurement meter.
    % In case you're wondering, the constants .778em and .322em specify
    % one quarter of a circle with radius 1.1em.
    % The points added at + and - <2.2em,2.2em> are there to strech the
    % canvas, ensuring that the size is unaffected by erratic spacing issues
    % with the arc.
\newcommand{\measure}[1]{*+[F-:<.9em>]{#1} \qw}
    % Inserts a measurement bubble with user defined text.
\newcommand{\measuretab}[1]{*{\xy*+<.6em>{#1}="e";"e"+UL;"e"+UR **\dir{-};"e"+DR **\dir{-};"e"+DL **\dir{-};"e"+LC-<.5em,0em> **\dir{-};"e"+UL **\dir{-} \endxy} \qw}
    % Inserts a measurement tab with user defined text.
\newcommand{\measureD}[1]{*{\xy*+=<0em,.1em>{#1}="e";"e"+UR+<0em,.25em>;"e"+UL+<-.5em,.25em> **\dir{-};"e"+DL+<-.5em,-.25em> **\dir{-};"e"+DR+<0em,-.25em> **\dir{-};{"e"+UR+<0em,.25em>\ellipse^{}};"e"+C:,+(0,1)*{} \endxy} \qw}
    % Inserts a D-shaped measurement gate with user defined text.
\newcommand{\multimeasure}[2]{*+<1em,.9em>{\hphantom{#2}} \qw \POS[0,0].[#1,0];p !C *{#2},p \drop\frm<.9em>{-}}
    % Draws a multiple qubit measurement bubble starting at the current position and spanning #1 additional gates below.
    % #2 gives the label for the gate.
    % You must use an argument of the same width as #2 in \ghost for the wires to connect properly on the lower lines.
\newcommand{\multimeasureD}[2]{*+<1em,.9em>{\hphantom{#2}} \POS [0,0]="i",[0,0].[#1,0]="e",!C *{#2},"e"+UR-<.8em,0em>;"e"+UL **\dir{-};"e"+DL **\dir{-};"e"+DR+<-.8em,0em> **\dir{-};{"e"+DR+<0em,.8em>\ellipse^{}};"e"+UR+<0em,-.8em> **\dir{-};{"e"+UR-<.8em,0em>\ellipse^{}},"i" \qw}
    % Draws a multiple qubit D-shaped measurement gate starting at the current position and spanning #1 additional gates below.
    % #2 gives the label for the gate.
    % You must use an argument of the same width as #2 in \ghost for the wires to connect properly on the lower lines.
\newcommand{\control}{*!<0em,.025em>-=-<.2em>{\bullet}}
    % Inserts an unconnected control.
\newcommand{\controlo}{*+<.01em>{\xy -<.095em>*\xycircle<.19em>{} \endxy}}
    % Inserts a unconnected control-on-0.
\newcommand{\ctrl}[1]{\control \qwx[#1] \qw}
    % Inserts a control and connects it to the object #1 wires below.
\newcommand{\ctrlo}[1]{\controlo \qwx[#1] \qw}
    % Inserts a control-on-0 and connects it to the object #1 wires below.
\newcommand{\targ}{*+<.02em,.02em>{\xy ="i","i"-<.39em,0em>;"i"+<.39em,0em> **\dir{-}, "i"-<0em,.39em>;"i"+<0em,.39em> **\dir{-},"i"*\xycircle<.4em>{} \endxy} \qw}
    % Inserts a CNOT target.
\newcommand{\qswap}{*=<0em>{\times} \qw}
    % Inserts half a swap gate.
    % Must be connected to the other swap with \qwx.
\newcommand{\multigate}[2]{*+<1em,.9em>{\hphantom{#2}} \POS [0,0]="i",[0,0].[#1,0]="e",!C *{#2},"e"+UR;"e"+UL **\dir{-};"e"+DL **\dir{-};"e"+DR **\dir{-};"e"+UR **\dir{-},"i" \qw}
    % Draws a multiple qubit gate starting at the current position and spanning #1 additional gates below.
    % #2 gives the label for the gate.
    % You must use an argument of the same width as #2 in \ghost for the wires to connect properly on the lower lines.
\newcommand{\ghost}[1]{*+<1em,.9em>{\hphantom{#1}} \qw}
    % Leaves space for \multigate on wires other than the one on which \multigate appears.  Without this command wires will cross your gate.
    % #1 should match the second argument in the corresponding \multigate.
\newcommand{\push}[1]{*{#1}}
    % Inserts #1, overriding the default that causes entries to have zero size.  This command takes the place of a gate.
    % Like a gate, it must precede any wire commands.
    % \push is useful for forcing columns apart.
    % NOTE: It might be useful to know that a gate is about 1.3 times the height of its contents.  I.e. \gate{M} is 1.3em tall.
    % WARNING: \push must appear before any wire commands and may not appear in an entry with a gate or label.
\newcommand{\gategroup}[6]{\POS"#1,#2"."#3,#2"."#1,#4"."#3,#4"!C*+<#5>\frm{#6}}
    % Constructs a box or bracket enclosing the square block spanning rows #1-#3 and columns=#2-#4.
    % The block is given a margin #5/2, so #5 should be a valid length.
    % #6 can take the following arguments -- or . or _\} or ^\} or \{ or \} or _) or ^) or ( or ) where the first two options yield dashed and
    % dotted boxes respectively, and the last eight options yield bottom, top, left, and right braces of the curly or normal variety.  See the Xy-pic reference manual for more options.
    % \gategroup can appear at the end of any gate entry, but it's good form to pick either the last entry or one of the corner gates.
    % BUG: \gategroup uses the four corner gates to determine the size of the bounding box.  Other gates may stick out of that box.  See \prop.

\newcommand{\rstick}[1]{*!L!<-.5em,0em>=<0em>{#1}}
    % Centers the left side of #1 in the cell.  Intended for lining up wire labels.  Note that non-gates have default size zero.
\newcommand{\lstick}[1]{*!R!<.5em,0em>=<0em>{#1}}
    % Centers the right side of #1 in the cell.  Intended for lining up wire labels.  Note that non-gates have default size zero.
\newcommand{\ustick}[1]{*!D!<0em,-.5em>=<0em>{#1}}
    % Centers the bottom of #1 in the cell.  Intended for lining up wire labels.  Note that non-gates have default size zero.
\newcommand{\dstick}[1]{*!U!<0em,.5em>=<0em>{#1}}
    % Centers the top of #1 in the cell.  Intended for lining up wire labels.  Note that non-gates have default size zero.
\newcommand{\Qcircuit}{\xymatrix @*=<0em>}
    % Defines \Qcircuit as an \xymatrix with entries of default size 0em.
\newcommand{\link}[2]{\ar @{-} [#1,#2]}
    % Draws a wire or connecting line to the element #1 rows down and #2 columns forward.
\newcommand{\pureghost}[1]{*+<1em,.9em>{\hphantom{#1}}}
    % Same as \ghost except it omits the wire leading to the left. 

\usepackage{graphicx,amsmath, amsthm, amssymb,color,url,booktabs,comment}  %cite
\urlstyle{sf}
%\usepackage[margin=1in]{geometry}
% \usepackage{fancyhdr}
\usepackage[colorlinks]{hyperref} %pagebackref

\newcommand{\nc}{\newcommand}
\nc{\rnc}{\renewcommand}
%
%\newcommand{\bra}[1]{\langle #1|}
%\newcommand{\ket}[1]{|#1\rangle}
\newcommand{\proj}[1]{|#1\rangle\langle #1|}
\newcommand{\braket}[2]{\langle #1|#2\rangle}
\newcommand{\Bra}[1]{\left\langle #1\right|}
\newcommand{\Ket}[1]{\left|#1\right\rangle}
\newcommand{\Proj}[1]{\left|#1\right\rangle\left\langle #1\right|}
\newcommand{\Braket}[2]{\left\langle #1\middle|#2\right\rangle}
\nc{\vev}[1]{\langle#1\rangle}
\nc{\grad}{{\vec{\nabla}}}
\DeclareMathOperator{\abs}{abs}
\DeclareMathOperator{\Bin}{Bin}
\DeclareMathOperator{\conv}{conv}
\DeclareMathOperator{\eig}{eig}
\DeclareMathOperator{\id}{id}
\DeclareMathOperator{\Img}{Im}
\DeclareMathOperator{\Par}{Par}
\DeclareMathOperator{\poly}{poly}
\DeclareMathOperator{\polylog}{polylog}
\DeclareMathOperator{\tr}{tr}
\DeclareMathOperator{\rank}{rank}
\DeclareMathOperator{\sgn}{sgn}
\DeclareMathOperator{\Sep}{Sep}
\DeclareMathOperator{\SepSym}{SepSym}
\DeclareMathOperator{\Span}{span}
\DeclareMathOperator{\supp}{supp}
\DeclareMathOperator{\swap}{SWAP}
\DeclareMathOperator{\Sym}{Sym}
\DeclareMathOperator{\ProdSym}{ProdSym}
\DeclareMathOperator{\SEP}{SEP}
\DeclareMathOperator{\PPT}{PPT}
\DeclareMathOperator{\Wg}{Wg}
\DeclareMathOperator{\WMEM}{WMEM}
\DeclareMathOperator{\WOPT}{WOPT}

\DeclareMathOperator{\BPP}{\mathsf{BPP}}
\DeclareMathOperator{\BQP}{\mathsf{BQP}}
\DeclareMathOperator{\cnot}{\normalfont\textsc{cnot}}
\DeclareMathOperator{\DTIME}{\mathsf{DTIME}}
\DeclareMathOperator{\NTIME}{\mathsf{NTIME}}
\DeclareMathOperator{\MA}{\mathsf{MA}}
\DeclareMathOperator{\NP}{\mathsf{NP}}
\DeclareMathOperator{\NEXP}{\mathsf{NEXP}}
\DeclareMathOperator{\Ptime}{\mathsf{P}}
\DeclareMathOperator{\QMA}{\mathsf{QMA}}
\DeclareMathOperator{\QCMA}{\mathsf{QCMA}}
\DeclareMathOperator{\BellQMA}{\mathsf{BellQMA}}

\newcommand{\be}{\begin{equation}}
\newcommand{\ee}{\end{equation}}
\newcommand{\bea}{\begin{eqnarray}}
\newcommand{\eea}{\end{eqnarray}}
\newcommand{\nn}{\nonumber}
\newcommand{\bi}{\begin{itemize}}
\newcommand{\ei}{\end{itemize}}
\newcommand{\bn}{\begin{enumerate}}
\newcommand{\en}{\end{enumerate}}
\def\beas#1\eeas{\begin{eqnarray*}#1\end{eqnarray*}}
\def\ba#1\ea{\begin{align}#1\end{align}}
\nc{\bas}{\[\begin{aligned}}
\nc{\eas}{\end{aligned}\]}
\nc{\bpm}{\begin{pmatrix}}
\nc{\epm}{\end{pmatrix}}
\def\non{\nonumber}
\def\nn{\nonumber}
\def\eq#1{(\ref{eq:#1})}
\def\eqs#1#2{(\ref{eq:#1}) and (\ref{eq:#2})}
%\def\eq#1{Eq.~(\ref{eq:#1})}
%\def\eqs#1#2{Eqs.~(\ref{eq:#1}) and (\ref{eq:#2})}
\def\L{\left} 
\def\R{\right}
\def\ra{\rightarrow}
\def\ot{\otimes}

%\newtheorem{thm}{Theorem}
%\newtheorem*{thm*}{Theorem}
%\newtheorem{claim}[thm]{Claim}
\newtheorem{cor}[thm]{Corollary}
%\newtheorem{lem}[thm]{Lemma}
%\newtheorem{prop}[thm]{Proposition}
%\newtheorem{dfn}{Definition}
%\newtheorem{proto}{Protocol}
\newtheorem{con}[thm]{Conjecture}

\makeatletter
\newtheorem*{rep@theorem}{\rep@title}
\newcommand{\newreptheorem}[2]{%
\newenvironment{rep#1}[1]{%
 \def\rep@title{#2 \ref{##1} (restatement)}%
 \begin{rep@theorem}}%
 {\end{rep@theorem}}}
\makeatother

%\newreptheorem{thm}{Theorem}
%\newreptheorem{lem}{Lemma}


\def\eps{\epsilon}
\def\va{{\vec{a}}}
\def\vb{{\vec{b}}}
\def\vn{{\vec{n}}}
\def\cvs{{\cdot\vec{\sigma}}}
\def\vx{{\vec{x}}}
\def\Usch{U_{\text{Sch}}}

\def\cA{\mathcal{A}}
\def\cB{\mathcal{B}}
\def\cD{\mathcal{D}}
\def\cE{\mathcal{E}}
\def\cF{\mathcal{F}}
\def\cH{\mathcal{H}}
\def\cI{{\cal I}}
\def\cL{{\cal L}}
\def\cM{{\cal M}}
\def\cN{\mathcal{N}}
\def\cO{{\cal O}}
\def\cP{\mathcal{P}}
\def\cQ{\mathcal{Q}}
\def\cS{\mathcal{S}}
\def\cT{{\cal T}}
\def\cU{\mathcal{U}}
\def\cW{{\cal W}}
\def\cX{{\cal X}}
\def\cY{{\cal Y}}

\def\bp{\mathbf{p}}
\def\bq{\mathbf{q}}
\def\bP{{\bf P}}
\def\bQ{{\bf Q}}
\def\gl{\mathfrak{gl}}

\def\bbC{\mathbb{C}}
\DeclareMathOperator*{\E}{\mathbb{E}}
\DeclareMathOperator*{\bbE}{\mathbb{E}}
\def\bbM{\mathbb{M}}
\def\bbN{\mathbb{N}}
\def\bbR{\mathbb{R}}
\def\bbZ{\mathbb{Z}}
\newcommand{\Real}{\textrm{Re}}

\def\benum{\begin{enumerate}}
\def\eenum{\end{enumerate}}
\def\bit{\begin{itemize}}
\def\eit{\end{itemize}}
\def\bdesc{\begin{description}}
\def\edesc{\end{description}}
\newcommand{\fig}[1]{Fig.~\ref{fig:#1}}
\newcommand{\tab}[1]{Table~\ref{tab:#1}}
\newcommand{\secref}[1]{Section~\ref{sec:#1}}
\newcommand{\appref}[1]{Appendix~\ref{sec:#1}}
\newcommand{\lemref}[1]{Lemma~\ref{lem:#1}}
\newcommand{\thmref}[1]{Theorem~\ref{thm:#1}}
\newcommand{\propref}[1]{Proposition~\ref{prop:#1}}
\newcommand{\protoref}[1]{Protocol~\ref{proto:#1}}
\newcommand{\defref}[1]{Definition~\ref{def:#1}}
\newcommand{\corref}[1]{Corollary~\ref{cor:#1}}
\newcommand{\conref}[1]{Conjecture~\ref{con:#1}}

\newcommand{\FIXME}[1]{{\color{red}FIXME: #1}}
\nc{\todo}[1]{\textcolor{red}{todo: #1}}



\newcommand{\boxdfn}[2]{
\begin{figure}[h]
\begin{center}
\noindent \framebox{
\begin{minipage}{0.8\textwidth}
\begin{dfn}[{\bf #1}]
\ \\ \\
#2
\end{dfn}
\end{minipage}
}
\end{center}
\end{figure}
}

\newcommand{\boxproto}[2]{
\begin{figure}[h]
\begin{center}
\noindent \framebox{
\begin{minipage}{0.8\textwidth}
\begin{proto}[{\bf #1}]
\ \\ \\
#2
\end{proto}
\end{minipage}
}
\end{center}
\end{figure}
}

\def\begsub#1#2\endsub{\begin{subequations}\label{eq:#1}\begin{align}#2\end{align}\end{subequations}}
\nc\qand{\qquad\text{and}\qquad}
\nc\mnb[1]{\medskip\noindent{\bf #1}}
\nc\mn{\medskip\noindent}

\renewcommand{\arraystretch}{1.5}
%\nc{\problem}[1]{\item\noindent {\bf #1}}

\setlength{\tabcolsep}{10pt}

%%%%%% Han-Hsuan's commands %%%%%%%%
\nc{\nl}{\nn \\ &=}  %new line
\nc{\nnl}{\nn \\ &}  %new new line
\nc{\fot}{\frac{1}{2}} %frac one two
\newcommand{\ben}{\begin{enumerate}}
\newcommand{\een}{\end{enumerate}}
\nc{\mc}{\mathcal}
\nc{\beq}{\begin{equation}}
\nc{\eeq}{\end{equation}}
\makeatother

\usepackage{babel}
  \providecommand{\algorithmname}{Algorithm}
  \providecommand{\definitionname}{Definition}
  \providecommand{\lemmaname}{Lemma}
  \providecommand{\problemname}{Problem}
  \providecommand{\remarkname}{Remark}
\providecommand{\theoremname}{Theorem}

\newcommand{\Expect}{{\rm I\kern-.3em E}}

\begin{document}

\title{The complexity of QMA with inverse exponential gap}
\author{Bill Fefferman, Cedric Lin}
\date{\today}
\maketitle
%\begin{abstract}
%
%\end{abstract}

We will study the complexity of QMA proof systems with inverse exponential gap. Define $\text{QMA}(c,s)$ to be QMA with completeness $c$ and soundness $s$, and write $\text{QMA}_{exp} := \cup_{c}\text{QMA}(c,c-2^{-\text{poly}})$. Our goal  will be to study the relationship between $\text{QMA}_{exp}$ and PSPACE.

\begin{con}
$\text{\emph{QMA}}_{exp}=\text{\emph{PSPACE}}$.
\end{con}

It is already known that
\begin{thm}[Ito, Kobayashi, and Watrous, Theorem 11] 
$\text{\emph{QMA}}(1,<1)\subseteq\text{\emph{PSPACE}}$.
\end{thm}
%Some other useful results:
%\begin{thm}[Watrous]
%$\text{\emph{PQP}} = \text{\emph{PP}}$, where \emph{PQP} is the unbounded-gap version of \emph{BQP}.
%\end{thm}
%\begin{thm}
%$\text{\emph{PQPSPACE}} = \text{\emph{PPSPACE}} = \text{\emph{PSPACE}} $.
%\end{thm}
%\begin{thm}[Aaronson]
%$\text{\emph{PostBQP}} = \text{\emph{PP}}$.
%\end{thm}
Another useful result:
\begin{thm}[Watrous]
$\text{\emph{PQPSPACE}} = \text{\emph{PPSPACE}} = \text{\emph{PSPACE}} $.
\end{thm}

\section{$\text{QMA}_{exp} \subseteq \text{PSPACE}$}
In this section we will sketch a proof of the following result:
\begin{thm}
$\text{\emph{QMA}}_{exp} \subseteq \text{\emph{PSPACE}}$.
\end{thm}
\begin{proof}
Let $L = (L_{yes},L_{no})$ be a promise problem in $\text{QMA}_m(c,s)$ with $c - s = 2^{-\text{poly}}$, where $m \in \text{poly}$ indicates the number of qubits in the witness. Suppose the collection of circuits $\{C'_x\}_x$ is a verification procedure for $L$; that is, each circuit $C'_x$ acts on $k+m$ qubits for some $k \in \text{poly}$, and if $x \in L_{yes}$ there exists an $m$-qubit state $\ket{\psi}$ such that
\begin{equation}
\L(\bra{\psi}\otimes \bra{0^k}\R) C'^\dagger_x \ket{1}\bra{1}_{out} C'_x \L(\ket{\psi}\otimes \ket{0^k}\R) \ge c
\end{equation}
whereas if $x \in L_{no}$, for all $m$-qubit states $\ket{\psi}$ we have
\begin{equation}
\L(\bra{\psi}\otimes \bra{0^k}\R) C'^\dagger_x \ket{1}\bra{1}_{out} C'_x \L(\ket{\psi}\otimes \ket{0^k}\R) \le s.
\end{equation}

We will use the following steps to decide which is the case in PSPACE.

\subsection{Gap Amplification}

Define the projectors
\begin{align}
\Pi_0 &= I_m \otimes \ket{0^k}\bra{0^k} \\
\Pi_1 &= C'^\dagger_x \L(\ket{1}\bra{1}_{out} \otimes I_{m+k-1}\R) C'_x
\end{align}
and the reflections
\begin{align}
R_0 &= 2\Pi_0 - I \\
R_1 &= 2\Pi_1 - I.
\end{align}
Note that we are asked to decide whether there exists $\ket{\psi}$ satisfying $\L(\bra{\psi}\otimes \bra{0^k}\R) \Pi_1 \L(\ket{\psi}\otimes \ket{0^k}\R) \ge c$, or that all states $\ket{\psi}$ satisfy $\L(\bra{\psi}\otimes \bra{0^k}\R) \Pi_1 \L(\ket{\psi}\otimes \ket{0^k}\R) \ge s$.
According to the fast QMA amplification procedure of Nagaj, Wocjan, and Zhang [arXiv:0904.1549], the following procedure will determine which is the case, with completeness at least $1-2^{-r}$ and soundness at most $2^{-r}$:

\begin{enumerate}
\item Perform $r$ trials of phase estimation of the operator $R_1R_0$ on the state $\ket{\psi}\otimes \ket{0^k}$, with $O(\log(c-s)) = O(\text{poly})$ bits of precision and $1/16$ failure probability. 
\item If the median of the $r$ results is at most $\phi_c = \arccos\sqrt{c}/\pi$, output YES; otherwise if the result is at least $\phi_s = \arccos\sqrt{s}/\pi$, output NO.
\end{enumerate}

Phase estimation of an operator $U$ up to $a$ bits of precision requires $O(a)$ ancilla qubits and $O(2^a)$ applications of the control-$U$ operation. Therefore this gap amplification procedure, for any polynomial $r$, requires an exponential number of gates to implement, but only a polynomial number of extra ancilla qubits.

We therefore see that we can obtain a (possibly exponentially long) circuit $C_x$, acting on $m+p$ qubits for some $p \in \text{poly}$, such that if $x \in L_{yes}$ there exists an $m$-qubit state $\ket{\psi}$ such that
\begin{equation}
\L(\bra{\psi}\otimes \bra{0^p}\R) C^\dagger_x \ket{1}\bra{1}_{out} C_x \L(\ket{\psi}\otimes \ket{0^p}\R) \ge 1-2^{-(m+2)}
\end{equation}
whereas if $x \in L_{no}$, for all $m$-qubit states $\ket{\psi}$ we have
\begin{equation}
\L(\bra{\psi}\otimes \bra{0^p}\R) C^\dagger_x \ket{1}\bra{1}_{out} C_x \L(\ket{\psi}\otimes \ket{0^p}\R) \le 2^{-(m+2)}.
\end{equation}
For convenience, define the $2^m \times 2^m$ matrix
\begin{equation}
Q_x := \L(I_{2^m}\otimes \bra{0^p}\R) C^\dagger_x \ket{1}\bra{1}_{out} C_x \L(I_{2^m}\otimes \ket{0^p}\R).
\end{equation}
Note that $Q_x$ is positive semidefinite, and $\bra{\psi}Q_x\ket{\psi}$ is the acceptance probability of our procedure on witness $\psi$ (at least $1-2^{-r}$ in the YES case, at most $2^{-r}$ in the NO case).

We use the following reduction from Marriott and Watrous: note that in the YES case $\tr[Q_x]\ge 1 - 2^{-m/2} \ge 3/4$, since the trace is at least the largest eigenvalue; while in the NO case $\tr[Q_x]\le 2^m \cdot 2^{-m/2} =1/4$ since the trace is the sum of the $2^m$ eigenvalues, each of which is at most $2^{-m/2}$. Therefore our problem reduces to determining whether the trace of $Q_x$ is at least $3/4$ or at most $1/4$.

We now see two different ways of proceeding from here, which we will detail in two separate subsections.

\subsection{Directly applying the $\text{PQPSPACE=PSPACE}$ relation}
We use the following idea, already present in the proof of Theorem 3.6 of Marriott and Watrous. Simply use the totally mixed state (alternatively, a random computational basis state) as the witness of the verification procedure encoded by $Q_x$. The totally mixed state is $2^{-m}I_m$, and therefore the acceptance probability is given by
\begin{equation}
\tr(Q_x 2^{-m}I_m) = 2^{-m} \tr(Q_x)
\end{equation}
which is at least $2^{-m} \cdot 3/4$ for the YES case, and at most $2^{-m} \cdot 1/4$ for the NO case. Thus we have reduced our original problem to determining whether an exponentially long quantum computation with \emph{no} witness, acting on a polynomial number of qubits, accepts with probability at least $c'$ or at most $s'$ with $c' - s'$ being exponentially small. This is a PQPSPACE (unbounded-error probabilitistic quantum polynomial space) problem; using Watrous's result that $\text{PQPSPACE=PSPACE}$ we immediately get that this is a PSPACE problem as well. Therefore $\text{QMA}_{exp} \subseteq \text{PSPACE}$, as claimed.

\subsection{Using path integrals}
We now present the other approach, which was our original idea. 

Write $\Delta_1 = \ket{1}\bra{1}_{out}$; then we want to calculate the quantity
\begin{equation}
\tr[Q_x] = \sum_{s_0 \in [2^m] \times \{0^p\}} \bra{s_0} C^\dagger_x \Delta_1 C_x \ket{s_0}.
\end{equation}
Let $C = U_TU_{T-1}\cdots U_1$ for single-qubit or two-qubit unitaries $U_i$; then
\begin{equation}
\tr[Q_x] = \sum_{s_0 \in [2^m] \times \{0^p\}} \bra{s_0} U_1^\dagger \cdots U_T^\dagger \Delta_1 U_T\cdots U_1 \ket{s_0}.
\end{equation}
Inserting complete sets of states $\ket{s'_1},\cdots,\ket{s'_T},\ket{s_T},\cdots,\ket{s_1}$ where $s'_i,s_i \in [2^{m+p}]$ we obtain
\begin{equation}
\tr[Q_x] = \sum_{s_0 \in [2^m] \times \{0^p\}} \sum_{s'_i,s_i \in [2^{m+p}]} \bra{s_0} U_1^\dagger \ket{s'_1}\bra{s'_1}U_2^\dagger\ket{s'_2} \cdots \bra{s'_{T-1}}U_T^\dagger\ket{s'_T} \bra{s'_T}\Delta_1 \ket{s_T}\bra{s_T}U_T\ket{s_{T-1}}\cdots \bra{s_1}U_1 \ket{s_0}.
\end{equation}
Since $T$ could be exponentially large, we see that $\tr[Q_x]$ is a sum over doubly exponentially many summands, each of which is a product over exponentially many factors. Nevertheless, by following the path integral approach of Fortnow and Rogers it should be possible to turn the calculation of $\tr[Q_x]$ into a GapPSPACE problem, and therefore determining whether $\tr[Q_x] \ge c'$ or $\le s'$ into a $\text{PPSPACE} = \text{PSPACE}$ problem. It should be clear how to do this if all the transition amplitudes in the unitaries $U_i$ are rational; the general case (with algebraic amplitudes) will take more work.

\end{proof}

\section{Lower bound}
We now discuss lower bounds for $\text{QMA}_{exp}$. The containment $\text{PP}\subseteq \text{QMA}_{exp}$ is fairly obvious: the completeness-soundness gap is at least inverse exponential for PP, and giving the verifier quantum power as well as a potential witness can only increase its power.

We will show something stronger:
\begin{thm}
$ \text{\emph{PSPACE}} \subseteq \text{\emph{QMA}}_{exp}$.
\end{thm}

\begin{proof}

Consider the problem of determining whether an exponentially-large, succinctly representable, sparse symmetric matrix is singular; we will show that this problem is PSPACE-complete. Note then that determining whether a sparse matrix $A$ is singular is equivalent to determining whether $A$ has 0 as an eigenvalue; we will show that the latter can be solved in $\text{QMA}_{exp}$.

By a matrix being succinctly representable and sparse, we mean the following:
\begin{defn}
Let $M$ be a $2^{\text{poly}(n)} \times 2^{\text{poly}(n)}$ matrix, where $n$ is the input size. We say that $M$ is a \emph{succinctly representable sparse} matrix if there are at most polynomially many nonzero entries in each row, and moreover there is a (uniformly generated) circuit to output the nonzero entries of any given row in $\text{poly}(n)$ time.
\end{defn}

\subsection{Sparse matrix singularity is PSPACE-complete}

We first state the following result of [arXiv:1210.1451]:
\begin{lem}
Let $A'$ be an exponentially-large, succinctly representable sparse matrix, whose determinant is promised to be 0, 1, or -1. Moreover, each column of $A'$ has at most two 1s. It is PSPACE-complete to determine if the determinant of $A'$ vanishes.
\end{lem}

We proceed to reproduce the proof of [arXiv:1210.1451] of this result, as we will need to modify it for our purposes.

\begin{proof}
Csanky's algorithm immediately gives that this problem is in PSPACE, and therefore we only need to show PSPACE-hardness.

Let $L=(L_{yes},L_{no}) \in \text{PSPACE}$ be decided by a deterministic Turing Machine $M$ in polynomial space. Consider the configuration graph $G^M$ of the matrix $M$: each vertex of $G_M$ corresponds to one of the exponentially many configurations of $M$, each of which is describable with polynomially many bits. The configuration graph $G^M$ has an edge from $c$ to $c'$ if and only if $c'$ can be reached from $c$ in one step of computation. It is straightforward to see that $G^M$ has the following properties:
\begin{itemize}
\item Since $M$ is deterministic, all vertices of $G^M$ have out-degree at most 1, and $G_M$ has no cycles.
\item The adjacency matrix of $G_M$ is a succinctly representable sparse matrix.
\item $M$ accepts input $x$ if and only if there is a path in $G^M$ from the starting configuration $s_x$ to the accepting configuration $t$.
\end{itemize}

Now on input $x$, consider the graph $G^M_x$ obtained by adding an edge from the accepting configuration $t$ to the starting configuration $s_x$, and adding self-loops on all other vertices. Let $A^M_x$ be the adjacency matrix of $G^M_x$. It then turns out that $\det(A^M_x) = \pm 1$ if and only if there is a path from $s_x$ to $t$, i.e. $M$ accepts input $x$; otherwise $\det(A^M_x) = 0$. Therefore computing $\det(A^M_x)$ is PSPACE-hard.
\end{proof}
We can immediately see the following:
\begin{cor}
Let $A$ be an exponentially-large, succinctly representable sparse matrix, whose determinant is promised to be 0, or 1. Moreover, $A$ is symmetric and positive semidefinite. It is PSPACE-complete to determine if the determinant of $A$ vanishes.
\end{cor}
\begin{proof}
Note that the matrix $(A^M_x)^T A^M_x$ is a succinctly representable sparse matrix, because there at most two 1s in each column of $A^M_x$; and deciding whether $\det((A^M_x)^T A^M_x) = \det(A^M_x)^2$ vanishes is PSPACE-hard.
\end{proof}

From here, we would like to argue that given a succinctly representable sparse and symmetric matrix $A$, it is PSPACE-complete to distinguish to determine whether $\det(A) = 0$ or $|\det(A)| > 2^{-\text{poly}}$, promised that one of this is the case. We will see later that this problem can be solved in $\text{QMA}_{exp}$. Unfortunately, this promise does not generally hold for succinctly representable sparse symmetric matrices; at worse if $A$ is nonsingular the smallest eigenvalue can still be doubly exponentially small. We will therefore need to modify the PSPACE-complete construction a bit.

First of all, we make the following definition:
\begin{defn}
A deterministic Turing machine $M$ is reversible if all vertices of the configuration graph $G^M$ have in-degree at most 1.
\end{defn}
Since we already know vertices of $G^M$ have out-degree at most 1, this implies that $G^M$ is simply a collection of disjoint paths.
\begin{thm}[Lange et al.]
Every bijective function computable by a deterministic Turing machine in space $S(n)$ can be computed by a reversible deterministic Turing machine in the same space $S(n)$.
\end{thm}
This is good news for us: it means it suffices to examine the case where $G^M$ is a collection of disjoint paths, and so the calculation of the eigenvalues of $G^M_x$ will be much simpler. In fact, [Lange] shows we can even assume the starting configuration $s_x$ has in-degree 0, as long as the input $x$ is kept on the tape at the end; the accepting configuration $t_x$ will therefore depend on $x$. 

Therefore if $\det(A^M_x) \neq 0$, i.e. if $M$ accepts $x$, there is a maximal path in $G^M$ starting from $s_x$ and ending at $t_x$. Assume the path has $\ell+1$ vertices, where $\ell = 2^{O(\text{poly})}$. $G^M_x$ adds an edge from $t_x$ to $s_x$ and adds a self-loop to all other vertices. Therefore if we reorder vertices and remove vertices not connected to $s_x$ and $t_x$, the relevant block of adjacency matrix $A^M_x$ is the $(\ell+1) \times (\ell+1)$ matrix
\begin{equation}
A'^M_x = 
\begin{bmatrix}
    0 & 0 & 0 & 0 & \dots  & 0  & 1 \\
    1 & 1 & 0 & 0 &\dots  & 0 & 0 \\
    0 & 1 & 1 & 0 & \dots  & 0 & 0 \\
     0 & 0 & 1 & 1 & \dots  & 0 & 0 \\
    \vdots & \vdots & \vdots & \vdots & \ddots & \vdots & \vdots \\
    0 & 0 & 0 & 0 & \dots  & 1 & 0 \\
    0 & 0 & 0 & 0 & \dots  & 1 & 0
\end{bmatrix}
\end{equation}
We can directly evaluate $(A'^M_x)^T A'^M_x$, obtaining the following matrix:
\begin{equation}
(A'^M_x)^T A'^M_x = 
\begin{bmatrix}
    1 & 1 & 0 & 0 & \dots  & 0 & 0  & 0 \\
    1 & 2 & 1 & 0 &\dots  & 0 & 0 & 0 \\
    0 & 1 & 2 & 1 & \dots  & 0 & 0 & 0 \\
     0 & 0 & 1 & 2 & \dots  & 0 & 0 & 0 \\
    \vdots & \vdots & \vdots & \vdots & \ddots & \vdots & \vdots \\
    0 & 0 & 0 & 0 & \dots  & 2 & 1 & 0 \\
    0 & 0 & 0 & 0 & \dots  & 1 & 2 & 0 \\
    0 & 0 & 0 & 0 & \dots  & 0 & 0 & 1
\end{bmatrix}
\end{equation}
For purposes of calculating the smallest eigenvalue the last row and column can be ignored, leaving an $\ell \times \ell$ matrix. The characteristic equation for the eigenvalues $\lambda$ is then $p_\ell(\lambda) = 0$, where
\begin{equation}
p_\ell(\lambda) := 
\det \L(
\begin{bmatrix}
    1 - \lambda & 1 & 0 & 0 & \dots  & 0 & 0  \\
    1 & 2 - \lambda & 1 & 0 &\dots  & 0 & 0 \\
    0 & 1 & 2 - \lambda & 1 & \dots  & 0 & 0 \\
     0 & 0 & 1 & 2 - \lambda & \dots  & 0 & 0 \\
    \vdots & \vdots & \vdots & \vdots & \ddots & \vdots \\
    0 & 0 & 0 & 0 & \dots  & 2 - \lambda & 1 \\
    0 & 0 & 0 & 0 & \dots  & 1 & 2 - \lambda \\
\end{bmatrix} \R).
\end{equation}
Now define the polynomial $q_n(x)$ to be the following $n \times n$ determinant:
\begin{equation}
q_n(x) := 
\det \L(
\begin{bmatrix}
    x & 1 & 0 & \dots  & 0 & 0  \\
    1 & x & 1 &\dots  & 0 & 0 \\
    0 & 1 & x  & \dots  & 0 & 0 \\
    \vdots & \vdots & \vdots & \ddots & \vdots \\
    0 & 0 & 0 & \dots  & x & 1 \\
    0 & 0 & 0 & \dots  & 1 & x \\
\end{bmatrix} \R)
\end{equation}
Note that $p_\ell(\lambda) = q_\ell(2-\lambda) - q_{\ell-1}(2-\lambda)$. Moreover, $q_n(x)$ satisfies the recurrence
\begin{equation}
q_0(x) = 1, \quad q_1(x) = x, \quad q_n(x) = x q_{n-1}(x) - q_{n-2} (x)
\end{equation}
This is the same recurrence satisfied by $U_n(x/2)$, where $U_n(x)$ is the $n$-th degree Chebyshev polynomial of the second kind. Therefore $q_n(x) = U_n(x/2)$. Moreover, using $U_n(\cos\theta) = \sin((n+1)\theta)/\sin(\theta)$ we can evaluate
\begin{align}
q_\ell(x) - q_{\ell}(x) &= \frac{\sin((\ell+1)\theta) - \sin(\ell \theta)}{\sin\theta} \\
&= \frac{\sin((2\ell+1)\theta/2)}{\sin (\theta/2)}
\end{align}
and therefore the zeroes of $q_\ell(x) - q_{\ell}(x)$ are $2\cos\L(\frac{2k}{2\ell + 1}\pi\R)$, $k = 1,\cdots,\ell$. The zeroes of the polynomial $p_\ell(\lambda) = q_\ell(2-\lambda) - q_{\ell-1}(2-\lambda)$ are therefore 
\begin{equation}
\lambda_k = 2\L(1 - \cos\L(\frac{2k}{2\ell+1}\pi\R)\R)
\end{equation}
and the smallest eigenvalue $\lambda_1 = \Theta(\ell^{-2})$ is inverse exponentially bounded away from zero, because $\ell = 2^{O(\text{poly})}$.

\subsection{Sparse matrix singularity is in $\text{QMA}_{exp}$}
We will now give a $\text{QMA}_{exp}$ algorithm for testing if the determinant of a succinctly representable sparse matrix $M$ vanishes, i.e. if 0 is an eigenvalue of $M$. First note the following:
\begin{rem}
Suppose a 0-1 matrix $M$ has at most $d$ nonzero entries per row. Then $\| M \| \le d$.
\end{rem}


\end{proof}

\end{document}
