\documentclass[11pt]{article}
\usepackage{fullpage}
\usepackage{times}
\usepackage{epsfig}
\usepackage{amsmath,amsthm,amsfonts,amssymb,amscd}

\usepackage{varioref}
\usepackage{verbatim} 
\usepackage{multicol}
\usepackage{lmodern}
\usepackage{enumerate}
%\usepackage[normalem]{ulem}
%
%\usepackage{caption}
%\usepackage{subcaption}
%\usepackage[T1]{fontenc}
%\usepackage[margin=1in]{geometry}
%\usepackage{fancyhdr}
%\usepackage{authblk}
%
%\usepackage{thm-restate}
%
\usepackage{mathrsfs}

\usepackage{hyperref}
\hypersetup{
    bookmarksnumbered=true, % If Acrobat bookmarks are requested, include section numbers
    unicode=false, % non-Latin characters in Acrobat bookmarks
    pdfstartview={}, % fits the width of the page to the window
    pdftitle={}, % title
    pdfauthor={}, % author
    pdfsubject={}, % subject of the document
    pdfcreator={}, % creator of the document
    pdfproducer={}, % producer of the document
    pdfkeywords={}, % list of keywords
    pdfnewwindow=true, % links in new window
    colorlinks=true, % false: boxed links; true: colored links
    linkcolor=blue, % color of internal links
    citecolor=blue, % color of links to bibliography
    filecolor=blue, % color of file links
    urlcolor=blue % color of external links
}

\newcommand{\classfont}{\sf}

\newtheorem{lemma}{Lemma}
\newtheorem{theorem}[lemma]{Theorem}
\newtheorem{corollary}[lemma]{Corollary}
\newtheorem{definition}[lemma]{Definition}
\newtheorem{proposition}[lemma]{Proposition}
\newtheorem{question}{Question}
\newtheorem{conjecture}{Conjecture}
\newtheorem{condition}{Condition}

\newtheorem{claim}{Claim}
\newtheorem{example}{Example}
\numberwithin{lemma}{section}
%\numberwithin{definition}{section}
%\numberwithin{question}{section} \numberwithin{claim}{lemma}
\newenvironment{restate}[1]{\bigskip \noindent {\bf #1 (restated).} \it}{\smallskip}

\theoremstyle{definition}
\newtheorem{remark}[lemma]{Remark}

\newcommand\PH{\sf{PH}}
\newcommand\BQP{\sf{BQP}}
\newcommand\NP{\sf{NP}}
\newcommand\MA{\sf{MA}}
\newcommand\AM{\sf{AM}}
\newcommand\PP{\sf{PP}}
\newcommand\BQSPACE{\sf{BQSPACE}}
\newcommand\PSPACE{\sf{PSPACE}}
\newcommand\Poly{\sf{P}}
\newcommand\EXP{{\sf{EXP}}}
\newcommand\DET{{\sf{DET}}}
\newcommand\Logspace{{\sf{L}}}
\newcommand\DSPACE{{\sf{SPACE}}}
\newcommand\NL{{\sf{NL}}}
\newcommand\preciseQMA{{\sf{preciseQMA}}}
\newcommand\QMA{{\sf{QMA}}}
\newcommand\QCMA{{\sf{QCMA}}}
\newcommand\expQCMA{{\sf{QCMA_{exp,poly}}}}
\newcommand\QCIP{{\sf{QCIP}}}
\newcommand\bigoh{\mathcal{O}}
\newcommand\PLclass{{\sf{PL}}}
\newcommand\C{{\mathbb{C}}}
\newcommand\R{{\mathbb{R}}}
\newcommand\Quat{{\mathbb{H}}}
\newcommand\M{{\mathbb{M}}}
\newcommand\F{{\mathbb{F}}}
\newcommand\Z{{\mathbb{Z}}}
\newcommand\N{\mathbb{N}}
\newcommand\Tr{{\mathop\textup{Tr }}}
\newcommand{\Ext}{E}
\newcommand{\Unitary}{\mathbf{U}}
\newcommand{\unitaryBQL}{{\classfont{BQL}}}
\newcommand{\unitaryQMASPACE}{{\classfont{QMA}_\Unitary\classfont{SPACE}}}
\newcommand{\unitaryQSPACE}[3]{{\classfont{Q}_\Unitary\classfont{SPACE}}[#1](#2,#3)}
\newcommand{\unitaryBQSPACE}[1]{{\classfont{BQ}_\Unitary\classfont{SPACE}}[#1]}

\newcommand{\rv}[1]{{\mathbf{#1}}}
\newcommand{\pr}[1]{\Pr\left[#1\right]}
\newcommand{\set}[1]{\left\{{#1}\right\}}
\newcommand{\poly}{\mathrm{poly}}
\newcommand{\remove}[1]{}

\renewcommand{\epsilon}{\varepsilon}
\newcommand{\eps}{\epsilon}
\newcommand*\samethanks[1][\value{footnote}]{\footnotemark[#1]}


\renewcommand{\leq}{\leqslant}
\renewcommand{\geq}{\geqslant}
\renewcommand{\le}{\leqslant}
\renewcommand{\ge}{\geqslant}
\newcommand{\eqdef}{\stackrel{{\rm def}}{=}}

\newcommand{\zo}{\{0,1\}}
\newcommand{\supp}{\mathrm{supp}}

\newcommand{\ee}{\mathcal{E}}
\newcommand{\ii}{\mathbb{I}}
\newcommand{\rl}{\rangle\langle}
\newcommand{\mg}{\mathcal{G}}
\newcommand{\hn}[1]{\|#1\|^H_{1\rightarrow 1}}
\newcommand{\ve}[1]{|#1\rangle\!\rangle}
\newcommand{\ro}[1]{\langle\!\langle#1|}
\newcommand{\bkett}[1]{|#1\rangle\!\rangle\langle\!\langle#1|}
\newcommand{\ket}[1]{|#1\rangle}
\newcommand{\bra}[1]{\langle#1|}
\newcommand{\bk}[1]{|#1\rangle\langle#1|}


\newcommand{\tth}[0]{\textsuperscript{th}}
\newcommand{\st}[0]{\textsuperscript{st}}
\newcommand{\nd}[0]{\textsuperscript{nd}}
\newcommand{\rd}[0]{\textsuperscript{rd}}


\usepackage{xcolor}
\newcommand{\todo}[1]{{\color{red}{[{\bf TODO:}#1]}}}
\newcommand{\skimmel}[1]{{\color{violet}{[{\bf skimmel:}#1]}}}
\newcommand{\wf}[1]{{\color{violet}{[{\bf wf:}#1]}}}
\newcommand{\tocite}[1]{{\color{blue}{[{\bf CITE:}#1]}}}
\newcommand{\tocheck}[1]{{\color{red}{[{\bf TO CHECK:}#1]}}}

\DeclareMathAlphabet{\matheu}{U}{eus}{m}{n}
\DeclareMathOperator*{\argmin}{arg\,min}
\DeclareMathOperator*{\argmax}{arg\,max}

\DeclareMathOperator{\tr}{tr}
\DeclareMathOperator{\id}{id}
\newcommand{\Favg}{\overline{F}}
\newcommand{\Paulis}{{\matheu P}}
\newcommand{\Clifs}{{\matheu C}}
\newcommand{\Hilb}{{\matheu H}}
\newcommand{\T}{{\matheu T}}
\newcommand{\sop}[1]{{\mathcal #1}}
\newcommand{\PL}[1]{{#1}^{P\!L}}
\newcommand{\BHn}{{{\mathcal B}({\mathcal H}^{\otimes n})}}
\newcommand{\I}{{\mathbb I}}
\newcommand{\CNOT}{{\mathrm{CNOT}}}
\newcommand{\plr}[1]{\hat{#1}} %pauli-liouville-represenation

%\newcommand{\ket}[1]{|{#1}\rangle}
%\newcommand{\bra}[1]{\langle{#1}|}
\newcommand{\braket}[2]{\langle{#1}|{#2}\rangle}
\newcommand{\ketbra}[2]{|{#1}\rangle\!\langle{#2}|}
\newcommand{\kket}[1]{|{#1}\rangle\!\rangle}
\newcommand{\bbra}[1]{\langle\!\langle{#1}|}
\newcommand{\bbrakket}[2]{\langle\!\langle{#1}|{#2}\rangle\!\rangle}

\newcommand{\no}{\nonumber\\}
\newcommand{\even}{_\textrm{even}}
\newcommand{\odd}{_\textrm{odd}}

\newcommand{\ns}{{\textsc{ns}}}
\newcommand{\ceil}[1]{\left\lceil{#1}\right\rceil}
\newcommand{\se}{\succcurlyeq}
\newcommand{\di}{\textrm{diag}}
\newcommand{\orcl}{{\pmb{\sop O}}}
\newcommand{\porcl}{{\pmb{\sop P}}}
\newcommand{\SA}{{\mathbf{S}_X^n}}
\newcommand{\SB}{{\mathbf{S}_Y^n}}
\newcommand{\Si}{{\mathbf{S}_1}}
\newcommand{\Sj}{{\mathbf{S}_2}}
\newcommand{\SU}{{S}}
\newcommand{\Ugroup}[1]{\mathbf{U}^{#1}}
\newcommand{\polyn}{\mathrm{poly\:}}
\usepackage{mathtools}
\usepackage{titling}
\newcommand\psham[1]{#1\textit{\sf{-Precise Succinct Hamiltonian}}}
\newcommand\matrixinversion[1]{#1\textit{\sf{-Well Conditioned Matrix Inversion}}}

\setlength{\droptitle}{-5em}

%\usepackage[hmargin=1.75cm,vmargin=1.0cm]{geometry}
\usepackage[hmargin=.85in,vmargin=.7in]{geometry}
\title{A Complete Characterization of Unitary Quantum Space \\ {\small \bf (3 page abstract)}}
\author{Bill Fefferman\thanks{Joint Center for Quantum Information and Computer Science (QuICS), University of Maryland,
College Park, MD 20742.}
\and {Cedric Lin\samethanks}}

\date{}
\begin{document}
\vspace{-2.0in}
\maketitle
\setcounter{page}{0}
\thispagestyle{empty}
\pagestyle{empty}
\vspace{-.6in}
\section{Introduction}
%My suggestion: we put all motivation in the introduction and then give proof sketches of the two main results in the next two sections
How powerful is quantum computation with a restricted number of qubits? In this work we give two complete problems for $\BQSPACE[k(n)]$, the class of problems that can be solved by a uniformly generated family of unitary\footnote{In this context a {\emph{unitary}} quantum circuit makes all measurements at the end of the computation.  Note that the standard method for deferring quantum measurements may incur an exponential blow-up in space.} quantum circuits that act on $k(n)$ qubits, for any fixed function $k$ between $\log{n}$ and $\polyn{n}$.

The first problem we consider, the $\psham{k(n)}$ problem, is a natural generalization of the familiar local Hamiltonian problem and asks us to decide if the ground state energy of a sparse Hamiltonian, whose entries are efficiently specified, is above or below a specified threshold.  As consequences we establish that $\QMA$ with exponentially small completeness-soundness gap is equal to $\PSPACE$, that determining whether a local Hamiltonian is frustration-free is $\PSPACE$-complete, and give a provable setting in which the ability to prepare $PEPS$ states gives less computational power than the ability to prepare the ground state of a generic local Hamiltonian.

The second problem we consider is a well-conditioned version of the ubiquitous matrix inversion problem.  Matrix inversion is known to be complete for $\DET$, the class of functions as hard as computing the determinant of an integer matrix, which can be solved in classical $\log^{2}(n)$ space. It is a major open problem to determine if Matrix Inversion can be solved in classical logarithmic space, which would imply $\Logspace=\NL=\DET$.  

Recently, Ta-Shma \cite{tashma}, building on work of Harrow, Hassidim, and Lloyd \cite{HHL}, showed that inverting a well-conditioned $n \times n$ matrix can be approximated by a quantum algorithm using $O(\log n)$ space, but with intermediate measurements. This gives a quadratic advantage in space over the best known classical algorithms, which require $\Omega(\log^2n)$ space.  This is the maximum quantum advantage possible, since Watrous has shown $\BQSPACE[k(n)]\subseteq\DSPACE[\bigoh(k(n)^2)]$ \cite{Watrous99,Watrous03} even for quantum algorithms with intermediate measurements. 
 
Our results improve upon Ta-Shma's result \cite{tashma} in two ways. First, unlike Ta-Shma's result, we show that this problem can be solved without intermediate measurements.  In addition, we show that the problem of inverting well-conditioned matrices is hard for unitary quantum logspace under $\Logspace$-reductions. In other words, one cannot show that well-conditioned matrices are invertible in $\Logspace$ unless one also shows that $\Logspace=\unitaryBQL$, which seems quite unlikely.

Furthermore, since it is straightforward to see that Well-conditioned Matrix Inversion reduces to Integer Matrix Inversion, we can use our result to obtain a direct proof that $\unitaryBQL\subseteq\DET$, which was previously known indirectly via the containments $\unitaryBQL\subseteq\PLclass\subseteq\DET$ \cite{Watrous03,Borodin84}.
     
    
\section{$\psham{k(n)}$}

\begin{definition}
[$\psham{k(n)}$ problem] \label{def: spechamiltonian}
Given as input is the size-$n$ efficient encoding of a $2^{k(n)} \times 2^{k(n)}$ PSD matrix $H$, such that % the maximum entry in absolute value 
 $\|H\|_{max} = \max_{s,t}|H(s,t)|$ is at most a constant. Let $\lambda_{min}$ be the minimum eigenvalue of $H$. It is promised that either $\lambda_{min} \le a$ or $\lambda_{min} \ge b$, where $a(n)$ and $b(n)$ are numbers such that $b-a > 2^{-\mathcal{O}(k(n))}$. Output 1 if $\lambda_{min} \le a$, and output 0 otherwise.
\end{definition}

\begin{theorem} \label{thm: spechamiltonian}
For $\Omega(\log(n)) \le k(n) \le \poly(n)$, $\psham{\mathcal{O}(k(n))}$ is complete for \\ $\unitaryBQSPACE{\mathcal{O}(k(n))}$ under classical reductions using $\poly(n)$ time and $\mathcal{O}(k(n))$ space.
\end{theorem}

\subsection{$\psham{k(n)}\in k(n)$-bounded $\preciseQMA$}
\subsection{$k(n)$-bounded $\preciseQMA\subseteq\BQSPACE[k(n)]$}
\section{$\matrixinversion{k(n)}$}
\begin{definition}[\matrixinversion{k(n)}] \label{def: matrix invert}
Given as input is the size-$n$ efficient encoding of a $2^{k(n)} \times 2^{k(n)}$ positive semidefinite matrix $H$ with a known upper bound $\kappa = 2^{\mathcal{O}(k(n))}$ on the condition number, so that $\kappa^{-1}I\preceq H \preceq I$, and $s,t\in \lbrace 0,1\rbrace^{k(n)}$. It is promised that either $|H^{-1}(s,t)|\geq b$
 or $|H^{-1}(s,t)|\leq a$ for some constants $0 \le a < b \le 1$; determine which is the case.
 \end{definition}
\begin{theorem} \label{thm: matrix invert}
For $\Omega(\log(n)) \le k(n) \le \poly(n)$, $\matrixinversion{\mathcal{O}(k(n))}$ is complete for $\BQSPACE{\mathcal{O}(k(n))}$ under classical reductions using $\poly(n)$ time and $\mathcal{O}(k(n))$ space.
\end{theorem}

%Here the subscript $\Unitary$ means that our quantum computation is unitary: there are no intermediate measurements. 
In particular, if we take $k = \bigoh(\log n)$ we get the following:
\begin{corollary}
The problem of approximating, to $1/\poly(n)$ precision, an entry of the inverse of an $n \times n$ PSD matrix with condition number at most $\poly(n)$ is $\unitaryBQL$-complete under $\Logspace$-reductions.
\end{corollary}

 \bibliographystyle{alpha}
\bibliography{completespace}
\end{document} 